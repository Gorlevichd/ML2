\documentclass[a4paper, 12pt]{article}
\usepackage{cmap}
\usepackage[T2A]{fontenc}
\usepackage[utf8]{inputenc}
\usepackage[russian, english]{babel}
\usepackage{statmath}
\usepackage{amsmath}
\usepackage{amssymb}

\title{МО2}
\date{March 2021}

\begin{document}

\maketitle

\section{Ядровые методы}

Данные: $x = (x_{1}, ... x_{m})$
\newline
Базисные функции: $\phi(x_{1}, ...)$
\newline
Модель принимает вид: $a(x) = \sum_{j = 1}^{m}w_{j}\phi_{j}(x)$
\newline
Для хорошего качества нужно много базисных функций $\rightarrow$ Ядровые методы позволяют не перебирать большое количество базисных функций

\begin{itemize}
    \item Быстрое обучение
\end{itemize}

\begin{center}
    \textit{Ядровые методы}
\end{center}
\begin{enumerate}
    \item Двойственное представление для линейной регрессии
    \newline
    
    $Q(w) = \frac{1}{2}\sum_{i = 1}^{l}(\sum_{j = 1}^{m}(w_{j}\phi_{j}(x_{i}) - y_{i})^{2} + \frac{\lambda}{2}||w||^{2}_{2} = \frac{1}{2}||\Phi w - y||_{2}^{2} + \frac{\lambda}{2}||w||_{2}^{2}$
    \newline
    
    $\Phi = \begin{pmatrix}\phi_{1}(x_{1}) & ... & \phi_{m}(x_{1}) \\ ... & ... & ... \\ \phi_{1}(x_{l}) & ... & \phi_{m}(x_{l}) \end{pmatrix}$
    
    $\nabla_{w}Q = \Phi^{T}(\Phi w - y) + \lambda w \rightarrow w = -\frac{1}{\lambda}\Phi^{T}(\Phi w - y) \rightarrow w = \Phi^{T}a$
    \newline
    
    w является линейной комбинацией строк $\Phi \rightarrow$ Решение можно искать из $w = \Phi^{T}a$ 
    
    $Q(a) = \frac{1}{2}||\Phi \Phi^{T}a - y|| + \frac{\lambda}{2}a^{T} \Phi \Phi^{T}a \rightarrow min_{a}$
    \newline
    
    $\Phi \Phi^{T}$ - матрица Грама (попарных скалярных произведений объектов)
    \newline
    
    Можно записать Q(w) так, что он зависит только от скалярных произведений объектов
    
    \item SVM
    
    $\begin{cases}
    \sum_{i = 1}^{l} \lambda_{i} - \frac{1}{2} \sum_{i, j = 1}^{l} \lambda_{i}\lambda_{j}y_{i}y_{j}<x_{i}, x_{j}> \rightarrow max_{\lambda} \\
    0 \geq \lambda_{i} \leq C
    \\
    \sum_{i = 1}^{l} \lambda_{i}y_{i} = 0
    \end{cases}$
    
    Такая формулировка задачи зависит от скалярных произведений объектов
    
    \item Алгоритм
    \begin{enumerate}
        \item Добавляем новые признаки
        \item Делаем это так, что $<\phi(x), \phi(z)>$ выражается через $<x, z>$
        \item Используем метод, который использует скалярные произведения объектов
        \item В этом методе $<x, z> \rightarrow <\phi(x), \phi(z)>$ (\textit{Kernel trick})
    \end{enumerate}
    \item \textbf{Ядро} - функция $K(x, z) = <\phi(x), \phi(z)>$, где $\phi: X \to H$
    \begin{enumerate}
        \item H - спрямляющее пространство
        \item $\phi$ - спрямляющее отображение
    \end{enumerate}
    
    \item \textbf{Теорема Мерсера}
    \begin{enumerate}
        \item $K(x, z) \text{ - ядро} \leftrightarrow \begin{cases}
        K(x, z) = K(z, x) \\ K \text{ неотрицательно определенная}
        \end{cases}$
        \item НО $ \rightarrow \forall l, \forall (x_{1}, ..., x_{l}) \in R^{d} \rightarrow (K(x_{i}, x_{j}))^{l}_{i, j = 1} \text{ НО}$
        \item \textit{На практике теорема Мерсера слишком сложна для применения}
    \end{enumerate}
    
    \item \textbf{Теорема 1}
    \begin{enumerate}
        \item Если
        \begin{enumerate}
            \item $K_{1}(x, z), K_{2}(x, z)$ - ядра, $x, z \in X$
            \item $f^{(x)}$ - вещественная функция на X
            \item $\phi: X \rightarrow R^{n}$
            \item $K_{3}$ - ядро заданное на $R^{n}$
        \end{enumerate}
        \item То \textit{cледующие функции являюися ядрами:}
        \begin{enumerate}
            \item $K(x, z) = K_{1}(x, z) + K_{2}(x, z)$
            \item $K(x, z) = \alpha K_{1}(x, z)$
            \item $K(x, z) = K_{1}K_{2}$
            \item $K(x, z) = f^{(x)}f^{(z)}$
            \item $K(x, z) = K(\phi(x), \phi(z))$
        \end{enumerate}
    \end{enumerate}
    \item \textbf{Теорема 2}
    \begin{enumerate}
        \item Если:
        \begin{enumerate}
                \item $K_{1}(x, z), K_{2}(x, z), ...$ - последовательность ядер
                \item $\exists K(x, z) = \lim_{n \to \infty}K_{n}(x, z), \forall x, z$
        \end{enumerate}
        \item То:
        \begin{enumerate}
            \item K - ядро
        \end{enumerate}
    \end{enumerate}
    \item \textbf{Полиномиальные ядра}
    \begin{enumerate}
        \item p(v) - многочлен с неотриц. коэфф
        \item $K(x, z) = w_{0} + w_{1}<x, z> + w_{2}<x, z>^{2} + ...$
        \item Является ядром по теореме 1
        \item $K(x, z) = (<x, z> + R)^{m} = \sum_{i = 0}^{m} C_{m}^{i}R^{m - i}<x, z>^{i}$
        \begin{enumerate}
            \item Если расписать все $<x, z>^{i}$, то получим все мономы степени i от исходных признаков 
            \item Зачем R? $\rightarrow$ коэффициент при мономе = $\sqrt{C_{m}^{i}R^{m - i}}$
            \item Сравним веса при мономах 1 и (m - 1)
            $\sqrt{\frac{C^{m - 1}_{m}R}{C^{1}_{m}R^{m-1}}} = \sqrt{\frac{1}{R^{m - 2}}}$
            \item R больше - мономы высоких степеней имеют низкий вклад
            \item Конечномерное спрямляющее пространство, но можно сделать линейно разделимое пространство
        \end{enumerate}
    \end{enumerate}
    \item \textbf{Гауссовы ядра}
    \begin{enumerate}
        \item Позволяет перевести в бесконечномерное спрямляющее пространство
        \item \fbox{$K(x, z) = exp\left(-\frac{||x - z||^{2}}{2\sigma^{2}}\right)$}
        \begin{enumerate}
            \item $exp(<x, z>) = \sum_{k = 0}^{\infty}\frac{<x, z>^{k}}{k\!}, \forall x, z = \lim_{n \to \infty}  \sum_{k = 0}^{\infty}\frac{<x, z>^{k}}{k\!}$
            \begin{enumerate}
                \item Разложение через ряд Тейлора
                \item Ядро, как последовательность ядер
            \end{enumerate}
            \item $\frac{exp(<x, z>)}{2\sigma^{2}}$ - ядро, аналогично
            \item $exp\left(-\frac{||x - z||^{2}}{2\sigma^{2}}\right) = exp\left(-\frac{<x - z, x - z>}{2\sigma^{2}}\right) = exp\left(-\frac{<x,x> -  <z, z>,  + <x, z>}{2\sigma^{2}}\right) = \frac{exp(<x, z> / \sigma^{2}}{exp(||x||^{2} / \sigma^{2})exp(||z||^{2} / \sigma^{2}})$
            \item $exp(<x, z> / \sigma^{2}) = K(x, z) = <\phi(x), \phi(z)>$
            \item $\tilde{\phi(x)} = \frac{\phi(x)}{||\phi(x)||} = \frac{\phi(x)}{\sqrt{K(x, x)}}$
            \item $<\tilde{\phi(x)}, \tilde{\phi(z)}> = \frac{<\phi(x), \phi(z)>}{\sqrt{K(x, x)K(z, z)}}$
        \end{enumerate}
        \item Какое спрямляющее пространство? - бесконечная сумма всех мономов
        \item \textit{Утверждение:} $x_{1}, ..., x_{l}$ - различные векторы из $\mathbb{R}^{d}$
        
        Тогда:
        
        $G = (exp\left(-\frac{||x - z||^{2}}{2\sigma^{2}}\right))^{l}_{i, j = 1}$ - невырожденная при $\sigma^{2} > 0$
        \item $x_{1}, ..., x_{l} \in \mathbb{R}^{d}$ - их матрица Грамма невырождена $\rightarrow \phi(x_{1}, ..., x_{l})$ ЛНЗ $\rightarrow$ бесконечное количество ЛНЗ векторов $\rightarrow$ бесконечномерное пространство
    \end{enumerate}
    \item \textbf{Ядровой SVM}
    \begin{enumerate}
        \item $ \begin{cases}
        \frac{1}{2}||w||^{2} + C\sum_{i = 1}^{l} \xi_{i} \rightarrow min_{w, b, \xi} \\
        y_{i}(<w, x_{i}> + b) \geq 1 - \xi_{i} \\
        \xi_{i} \geq 0
        \end{cases} $
        \[L(w, b, \xi, \lambda, \mu) = \frac{1}{2}||w||^2 + C\sum_{i = 1}^{d}\xi_i - \sum_{i = 1}^{l}\lambda_i (y_i(<w, x_i> + b) - 1 + \xi_i) - \sum_{i = 1}^l \mu_i \xi_i\]
        
        В точке оптимума $\nabla_w L = 0$
        
        \[\nabla_w L = w - \sum_{i = 1}^l \lambda y_i x_i = 0 \rightarrow w = \sum_{i = 1}^l \lambda_i y_i x_i\]
        \[\nabla_b L = \sum_{i = 1}^{l} \lambda_i y_i = 0\]
        \[\nabla_{\xi_i} L = C - \lambda_i - \mu_i = 0 \rightarrow \lambda_i + \mu_i = C\]
        
        Условие дополняющей нежесткости:
        
        \[\lambda_i (y_i(<w, x_i> + b) - 1 + \xi_i) = 0 \rightarrow \lambda_i = 0 \textrm{ или } (y_i(<w, x_i> + b) - 1 + \xi_i)\]
        \[\mu_i \xi_i = 0 \rightarrow \mu_i = 0 \textrm{ или } \xi_i = 0\]

        Свойства лагранжиана:

        \[\lambda \geq 0, \mu \geq 0\]

        \item Типы объектов
        \begin{enumerate}
            \item $\lambda_i = 0 \rightarrow \mu_i = C \rightarrow \xi_i = 0 \rightarrow x_i$ лежит с правильной стороны от 
            разделяющей гиперплоскости и на достаточном расстоянии от нее. $w = \sum_{i = 1}^l \lambda y_i x_i \rightarrow$ объект не влияет на веса. 
            Называется \textbf{периферийный.}
            \item $0 < \lambda_i < 1 \rightarrow \mu \neq 0 \rightarrow \xi_0 = 0$. $x_i$ не залезает на разделяющую полосу, но $y_i(<w, x_i> + b) = 1 \rightarrow x_i$ лежит прямо на границе.
            Дает вклад в $w$. $x_i$ - \textbf{опорный граничный.}
            \item $\lambda_i = C \rightarrow \xi_i > 0$. 
            $x_i$ дает вклад в w. $\xi_i > 0 \rightarrow x_i$ нарушает границу - \textbf{Опорные нарушители}.      
        \end{enumerate}
        \item Подставляем $w = \sum_{i = 1}^l \lambda y_i x_i$ в лагранжиан, учтем ограничения $\sum_{i = 1}^l \lambda_i y_i = 0$ и $C - \lambda_i - \mu_i = 0$
        
        \textbf{Двойственная задача SVM}
        \[\begin{cases}
            L = \sum_{i = 1}^l \lambda_i - \frac{1}{2}\sum_{i, j = 1}^l \lambda_i \lambda_j y_i y_j <x_i, x_j> \rightarrow max_{\lambda} \\
            \sum_{i = 1}^l \lambda_i y_i = 0 \\
            0 \leq \lambda_i \leq C
        \end{cases}\]

        \item Если $\lambda$ - решение, то $w = \sum_{i = 1}^l \lambda_i y_i x_i$ - решение исходной задачи
        \item Задача зависит от объектов только через скалярное произведение $\rightarrow$
        можно заменить его на ядро
        \item Находим b
        Берем $x_i: 0 < \lambda_i < C \rightarrow \xi_i = 0 \rightarrow y_i(<w, x_i> + b) = 1 \rightarrow b = y_i - <w, x_i>$
        \item Минусы ядрового SVM
        \begin{enumerate}
            \item Сложно контролировать переобучение
            \item Необходимо хранить в памяти матрицу Грамма
            \item Нельзя менять функцию потерь
        \end{enumerate}
    \end{enumerate}
    \item \textbf{Применение ядерной модели}
    \begin{enumerate}
        \item $a(x) = sign(<w, x> + b) = sign(<\sum_{i = 1}^l \lambda y_i x_i, x> + b) = sign(\sum_{i = 1}^l \lambda_i y_i <x_i, x> + b)$ 
    \end{enumerate}
\end{enumerate}
\end{document}
