\documentclass[a4paper, 12pt]{article}
\usepackage{cmap}
\usepackage[T2A]{fontenc}
\usepackage[utf8]{inputenc}
\usepackage[russian, english]{babel}
\usepackage{statmath}
\usepackage{amsmath}
\usepackage{amssymb}
\usepackage{listings}
\setlength\parindent{0pt}
\catcode`@=11
\def\caseswithdelim#1#2{\left#1\,\vcenter{\normalbaselines\m@th
  \ialign{\strut$##\hfil$&\quad##\hfil\crcr#2\crcr}}\right.}% you might like it without the \strut
\catcode`@=12
%
\def\bcases#1{\caseswithdelim[{#1}}
\def\vcases#1{\caseswithdelim|{#1}}
%

\newcommand\norm[1]{\left\lVert#1\right\rVert}

\title{МО2}
\date{March 2021}

\begin{document}

\maketitle

\part{Лекции}

\section{Ядровые методы}

Данные: $x = (x_{1}, ... x_{m})$
\newline
Базисные функции: $\phi(x_{1}, ...)$
\newline
Модель принимает вид: $a(x) = \sum_{j = 1}^{m}w_{j}\phi_{j}(x)$
\newline
Для хорошего качества нужно много базисных функций $\rightarrow$ Ядровые методы позволяют не перебирать большое количество базисных функций

\begin{itemize}
    \item Быстрое обучение
\end{itemize}

\begin{center}
    \textit{Ядровые методы}
\end{center}
\begin{enumerate}
    \item Двойственное представление для линейной регрессии
    \newline
    
    $Q(w) = \frac{1}{2}\sum_{i = 1}^{l}(\sum_{j = 1}^{m}(w_{j}\phi_{j}(x_{i}) - y_{i})^{2} + \frac{\lambda}{2}||w||^{2}_{2} = \frac{1}{2}||\Phi w - y||_{2}^{2} + \frac{\lambda}{2}||w||_{2}^{2}$
    \newline
    
    $\Phi = \begin{pmatrix}\phi_{1}(x_{1}) & ... & \phi_{m}(x_{1}) \\ ... & ... & ... \\ \phi_{1}(x_{l}) & ... & \phi_{m}(x_{l}) \end{pmatrix}$
    
    $\nabla_{w}Q = \Phi^{T}(\Phi w - y) + \lambda w \rightarrow w = -\frac{1}{\lambda}\Phi^{T}(\Phi w - y) \rightarrow w = \Phi^{T}a$
    \newline
    
    w является линейной комбинацией строк $\Phi \rightarrow$ Решение можно искать из $w = \Phi^{T}a$ 
    
    $Q(a) = \frac{1}{2}||\Phi \Phi^{T}a - y|| + \frac{\lambda}{2}a^{T} \Phi \Phi^{T}a \rightarrow min_{a}$
    \newline
    
    $\Phi \Phi^{T}$ - матрица Грама (попарных скалярных произведений объектов)
    \newline
    
    Можно записать Q(w) так, что он зависит только от скалярных произведений объектов
    
    \item SVM
    
    $\begin{cases}
    \sum_{i = 1}^{l} \lambda_{i} - \frac{1}{2} \sum_{i, j = 1}^{l} \lambda_{i}\lambda_{j}y_{i}y_{j}<x_{i}, x_{j}> \rightarrow \max_{\lambda} \\
    0 \geq \lambda_{i} \leq C
    \\
    \sum_{i = 1}^{l} \lambda_{i}y_{i} = 0
    \end{cases}$
    
    Такая формулировка задачи зависит от скалярных произведений объектов
    
    \item Алгоритм
    \begin{enumerate}
        \item Добавляем новые признаки
        \item $x, z \in X$
        \item Делаем это так, что $<\phi(x), \phi(z)>$ выражается через $<x, z>$
        \item Используем метод, который использует скалярные произведения объектов
        \item В этом методе $<x, z> \rightarrow <\phi(x), \phi(z)>$ (\textit{Kernel trick})
    \end{enumerate}
    \item \textbf{Ядро} - функция $K(x, z) = <\phi(x), \phi(z)>$, где $\phi: X \to H$
    \begin{enumerate}
        \item H - спрямляющее пространство
        \item $\phi$ - спрямляющее отображение
    \end{enumerate}
    
    \item \textbf{Теорема Мерсера}
    \begin{enumerate}
        \item $K(x, z) \text{ - ядро} \leftrightarrow \begin{cases}
        K(x, z) = K(z, x) \\ K \text{ неотрицательно определенная}
        \end{cases}$
        \item НО $ \rightarrow \forall l, \forall (x_{1}, ..., x_{l}) \in R^{d} \rightarrow (K(x_{i}, x_{j}))^{l}_{i, j = 1} \text{ НО}$
        \item \textit{На практике теорема Мерсера слишком сложна для применения}
    \end{enumerate}
    
    \item \textbf{Теорема 1}
    \begin{enumerate}
        \item Если
        \begin{enumerate}
            \item $K_{1}(x, z), K_{2}(x, z)$ - ядра, $x, z \in X$
            \item $f^{(x)}$ - вещественная функция на X
            \item $\phi: X \rightarrow R^{n}$
            \item $K_{3}$ - ядро заданное на $R^{n}$
        \end{enumerate}
        \item То \textit{cледующие функции являюися ядрами:}
        \begin{enumerate}
            \item $K(x, z) = K_{1}(x, z) + K_{2}(x, z)$
            \item $K(x, z) = \alpha K_{1}(x, z)$
            \item $K(x, z) = K_{1}K_{2}$
            \item $K(x, z) = f^{(x)}f^{(z)}$
            \item $K(x, z) = K(\phi(x), \phi(z))$
        \end{enumerate}
    \end{enumerate}
    \item \textbf{Теорема 2}
    \begin{enumerate}
        \item Если:
        \begin{enumerate}
                \item $K_{1}(x, z), K_{2}(x, z), ...$ - последовательность ядер
                \item $\exists K(x, z) = \lim_{n \to \infty}K_{n}(x, z), \forall x, z$
        \end{enumerate}
        \item То:
        \begin{enumerate}
            \item K - ядро
        \end{enumerate}
    \end{enumerate}
    \item \textbf{Полиномиальные ядра}
    \begin{enumerate}
        \item p(v) - многочлен с неотриц. коэфф
        \item $K(x, z) = w_{0} + w_{1}<x, z> + w_{2}<x, z>^{2} + ...$
        \item Является ядром по теореме 1
        \item $K(x, z) = (<x, z> + R)^{m} = \sum_{i = 0}^{m} C_{m}^{i}R^{m - i}<x, z>^{i}$
        \begin{enumerate}
            \item Если расписать все $<x, z>^{i}$, то получим все мономы степени i от исходных признаков 
            \item Зачем R? $\rightarrow$ коэффициент при мономе = $\sqrt{C_{m}^{i}R^{m - i}}$
            \item Сравним веса при мономах 1 и (m - 1)
            $\sqrt{\frac{C^{m - 1}_{m}R}{C^{1}_{m}R^{m-1}}} = \sqrt{\frac{1}{R^{m - 2}}}$
            \item R больше - мономы высоких степеней имеют низкий вклад
            \item Конечномерное спрямляющее пространство, но можно сделать линейно разделимое пространство
        \end{enumerate}
    \end{enumerate}
    \item \textbf{Гауссовы ядра}
    \begin{enumerate}
        \item Позволяет перевести в бесконечномерное спрямляющее пространство
        \item \fbox{$K(x, z) = exp\left(-\frac{||x - z||^{2}}{2\sigma^{2}}\right)$}
        \begin{enumerate}
            \item $exp(<x, z>) = \sum_{k = 0}^{\infty}\frac{<x, z>^{k}}{k\!}, \forall x, z = \lim_{n \to \infty}  \sum_{k = 0}^{\infty}\frac{<x, z>^{k}}{k\!}$
            \begin{enumerate}
                \item Разложение через ряд Тейлора
                \item Ядро, как последовательность ядер
            \end{enumerate}
            \item $\frac{exp(<x, z>)}{2\sigma^{2}}$ - ядро, аналогично
            \item $exp\left(-\frac{||x - z||^{2}}{2\sigma^{2}}\right) = exp\left(-\frac{<x - z, x - z>}{2\sigma^{2}}\right) = exp\left(-\frac{<x,x> -  <z, z>,  + <x, z>}{2\sigma^{2}}\right) = \frac{exp(<x, z> / \sigma^{2}}{exp(||x||^{2} / \sigma^{2})exp(||z||^{2} / \sigma^{2}})$
            \item $exp(<x, z> / \sigma^{2}) = K(x, z) = <\phi(x), \phi(z)>$
            \item $\tilde{\phi(x)} = \frac{\phi(x)}{||\phi(x)||} = \frac{\phi(x)}{\sqrt{K(x, x)}}$
            \item $<\tilde{\phi(x)}, \tilde{\phi(z)}> = \frac{<\phi(x), \phi(z)>}{\sqrt{K(x, x)K(z, z)}}$
        \end{enumerate}
        \item Какое спрямляющее пространство? - бесконечная сумма всех мономов
        \item \textit{Утверждение:} $x_{1}, ..., x_{l}$ - различные векторы из $\mathbb{R}^{d}$
        
        Тогда:
        
        $G = (exp\left(-\frac{||x - z||^{2}}{2\sigma^{2}}\right))^{l}_{i, j = 1}$ - невырожденная при $\sigma^{2} > 0$
        \item $x_{1}, ..., x_{l} \in \mathbb{R}^{d}$ - их матрица Грамма невырождена $\rightarrow \phi(x_{1}, ..., x_{l})$ ЛНЗ $\rightarrow$ бесконечное количество ЛНЗ векторов $\rightarrow$ бесконечномерное пространство
    \end{enumerate}
    \item \textbf{Ядровой SVM}
    \begin{enumerate}
        \item $ \begin{cases}
        \frac{1}{2}||w||^{2} + C\sum_{i = 1}^{l} \xi_{i} \rightarrow min_{w, b, \xi} \\
        y_{i}(<w, x_{i}> + b) \geq 1 - \xi_{i} \\
        \xi_{i} \geq 0
        \end{cases} $
        \[L(w, b, \xi, \lambda, \mu) = \frac{1}{2}||w||^2 + C\sum_{i = 1}^{d}\xi_i - \sum_{i = 1}^{l}\lambda_i (y_i(<w, x_i> + b) - 1 + \xi_i) - \sum_{i = 1}^l \mu_i \xi_i\]
        
        В точке оптимума $\nabla_w L = 0$
        
        \[\nabla_w L = w - \sum_{i = 1}^l \lambda y_i x_i = 0 \rightarrow w = \sum_{i = 1}^l \lambda_i y_i x_i\]
        \[\nabla_b L = \sum_{i = 1}^{l} \lambda_i y_i = 0\]
        \[\nabla_{\xi_i} L = C - \lambda_i - \mu_i = 0 \rightarrow \lambda_i + \mu_i = C\]
        
        Условие дополняющей нежесткости:
        
        \[\lambda_i (y_i(<w, x_i> + b) - 1 + \xi_i) = 0 \rightarrow \lambda_i = 0 \textrm{ или } (y_i(<w, x_i> + b) - 1 + \xi_i) =0\]
        \[\mu_i \xi_i = 0 \rightarrow \mu_i = 0 \textrm{ или } \xi_i = 0\]

        Свойства лагранжиана:

        \[\lambda \geq 0, \mu \geq 0\]

        \item Типы объектов
        \begin{enumerate}
            \item $\lambda_i = 0 \rightarrow \mu_i = C \rightarrow \xi_i = 0 \rightarrow x_i$ лежит с правильной стороны от 
            разделяющей гиперплоскости и на достаточном расстоянии от нее. $w = \sum_{i = 1}^l \lambda y_i x_i \rightarrow$ объект не влияет на веса. 
            Называется \textbf{периферийный.}
            \item $0 < \lambda_i < 1 \rightarrow \mu \neq 0 \rightarrow \xi_0 = 0$. $x_i$ не залезает на разделяющую полосу, но $y_i(<w, x_i> + b) = 1 \rightarrow x_i$ лежит прямо на границе.
            Дает вклад в $w$. $x_i$ - \textbf{опорный граничный.}
            \item $\lambda_i = C \rightarrow \xi_i > 0$. 
            $x_i$ дает вклад в w. $\xi_i > 0 \rightarrow x_i$ нарушает границу - \textbf{Опорные нарушители}.      
        \end{enumerate}
        \item Подставляем $w = \sum_{i = 1}^l \lambda y_i x_i$ в лагранжиан, учтем ограничения $\sum_{i = 1}^l \lambda_i y_i = 0$ и $C - \lambda_i - \mu_i = 0$
        
        \textbf{Двойственная задача SVM}
        \[\begin{cases}
            L = \sum_{i = 1}^l \lambda_i - \frac{1}{2}\sum_{i, j = 1}^l \lambda_i \lambda_j y_i y_j <x_i, x_j> \rightarrow \max_{\lambda} \\
            \sum_{i = 1}^l \lambda_i y_i = 0 \\
            0 \leq \lambda_i \leq C
        \end{cases}\]

        \item Если $\lambda$ - решение, то $w = \sum_{i = 1}^l \lambda_i y_i x_i$ - решение исходной задачи
        \item Задача зависит от объектов только через скалярное произведение $\rightarrow$
        можно заменить его на ядро
        \item Находим b
        Берем $x_i: 0 < \lambda_i < C \rightarrow \xi_i = 0 \rightarrow y_i(<w, x_i> + b) = 1 \rightarrow b = y_i - <w, x_i>$
        \item Минусы ядрового SVM
        \begin{enumerate}
            \item Сложно контролировать переобучение
            \item Необходимо хранить в памяти матрицу Грамма
            \item Нельзя менять функцию потерь
        \end{enumerate}
    \end{enumerate}
    \item \textbf{Применение ядерной модели}
    \begin{enumerate}
        \item $a(x) = sign(<w, x> + b) = sign(<\sum_{i = 1}^l \lambda y_i x_i, x> + b) = sign(\sum_{i = 1}^l \lambda_i y_i <x_i, x> + b)$ 
    \end{enumerate}
\end{enumerate}

\section{Аппроксимации ядер, EM алгоритм}

Скалярные произведения тяжело хранить из-за
размера матрицы. 

Есть ли возможность построить 
$\tilde{\phi}(x) \rightarrow 
<\tilde{\phi}(x_i), \tilde{\phi}(x_j)> 
\approx K(x_i, x_j)$

\subsection{Метод случайных признаков Фурье}

\[K(x, z) = K(x - z)\]

K - непрерывная функция

\textit{Теорема Бохнера}

\[K(x - z) \rightarrow \exists p(w) \rightarrow 
K(x - z) = \int_{R^{d}} p(w)e^{iw^{T}(x - z)}dw\]

\textit{Используем:}

\[K(x - z) = \int_{R^{d}} p(w)e^{iw^{T}(x - z)}dw 
\xrightarrow{\textrm{Формула Эйлера \footnotemark[1]}} 
\int_{R^{d}} p(w)cos(w^{T}(x - z)) + i\int_{R^{d}} p(w)cos(w^{T}(x - z))\] 
\[\xrightarrow{K(x - z) \textrm{ - веществ.}} 
\textrm{ Комплексная часть = 0} \rightarrow 
K(x - z) = \int_{R^{d}} p(w)cos(w^{T}(x - z))dw\]
\[\xrightarrow{ \textrm{Монте-Карло \footnotemark[2]}}
K(x - z) \approx\{w_j \sim p(w)\}: 
\frac{1}{n} \sum_{i = 1}^{n} cos w_j^T(x - z)\]
\[= \frac{1}{n} \sum_{i = 1}^n cos w_j^Tx cos w_j^Tz + sin w_j^Tx sin w_j^Tz\]
\footnotetext[1]{$e^{ix} = cosx + isinx$}
\footnotetext[2]{$\int_{a}^{b}f(x)dx = \frac{b - a}{n} \sum_{i = 1}^N f(u_i)$}

\[\tilde{\phi}(x) = \frac{1}{n}(cos w_1^Tx, \ldots, cos w_n^Tx, sin w_1^Tx, \ldots, sin w_n^Tx)\]
\[K(x - z) = <\tilde{\phi}(x), \tilde{\phi}(z)>\]

Для гауссова ядра:

\[p(w) = \mathcal{N}(0, 1)\]

\subsection{EM алгоритм}

Смесь распределений: 

\[\begin{cases}
    p(x) = \sum_{k = 0}^{K} \pi_k p_k(x) \\
    \sum \pi_k = 1    
\end{cases}
\]

Вероятностный эксперимент:

Выбираем K из $[\pi_1, \ldots, \pi_K]$, выбираем x из $pi_k(x)$

Z - скрытые переменные

\[Z = \{0, 1\}^K, \sum Z_k = 1\]
\[p(Z_k = 1) = \pi_k\]
\[p(z) = \prod_{k = 1}^K \pi_k^{Z_k}\]
\[p(x \mid Z_k = 1) = p_k(x)\]
\[p(x \mid z) = \prod_{k = 1}^{K} (p_{k}(x)^{Z_{k}})\]
\[p(x, z) = p(x \mid z)p(z) = \prod_{k = 1}^{K} (\pi_k p_k(x))^{Z_{k}}\]
\[p(x) = \sum_{k = 1}^{K} p(x, z = k) = \sum_{k = 1}^{K} \pi_k p_k(x)\]

Вероятностная кластеризация:

\(p_k(x)\) - распределение k-го кластера

\[x \rightarrow (p_1(x), \ldots, p_k(x))\]

Хотим описать X смесью распределений

\[p(x) = \sum_{k = 1}^K \pi_k \phi(x \mid \theta_k), \phi(x \mid \theta_k) \sim \mathcal{N}(\mu, \Sigma), \theta = (\mu, \Sigma)\]

Неполное правдоподобие:

\[ln(P(X \mid \Theta)) = \sum_{i = 1}^{l} log \sum_{k = 1}^K \pi_k \phi(x_i \mid \theta_k) \rightarrow max_{\theta}\]

Логарифм многооптимальная функция - просто оптимизировать ее сложно

Используем функцию полного правдоподобия

\[log(P, X \mid \Theta) = \sum_{i = 1}^{l} log \sum_{k = 1}^{K} (\pi_k \phi(x_i \mid \theta_k))^{Z_k}\]
\[\sum_{i = 1}^{l}\sum_{k = 1}^{K} Z_{ik}(log\pi_k + log\phi(x_i \mid \theta_k)) \rightarrow \max_{\Theta}\]

Известно аналитическое решение для нормального распределения.

Не знаем $Z_ik$

\[\Theta = (\pi_1, \ldots, \pi_k, \theta_1, \ldots, \theta_k)\]

Используем метод ALS для поиска $Z, \Theta$

\begin{enumerate}
    \item Оптимизация по скрытым переменным
    
    Апостериорное распределение: \(p(Z \mid X, \Theta) = \frac{P(X, Z \mid \Theta)}{p(X \mid \Theta)}\)

    \[Z^{\star} = \argmax_{Z} p(Z \mid X, \Theta)\]

    \item Оптимизировать по $\Theta$
    
    \[log p(X, Z^{\star} \mid \Theta) \rightarrow \max_{\Theta}\]
    \item Повторять до сходимости
\end{enumerate}

Можно лучше. Не гарантирует сходимости


EM-алгоритм - метод обучения моделей со скрытыми переменными

\textbf{EM-алгоритм}

\begin{enumerate}
    \item E-шаг - вычисляем $p(Z \mid X, \Theta)$ и запоминаем
    \item M-шаг
    
    \[E_{Z \sim p(Z \mid X, \Theta)}log p(X, Z \mid \Theta) = 
    \sum_{Z} p(Z \mid X, \Theta)log p(X, Z \mid \Theta) 
    \rightarrow \max_{\Theta}\]
\end{enumerate}

Вывод EM-алгоритма

\[log p(X \mid \Theta) = Z(q, \Theta) + KL(q \mid \mid p)\]

\[L(q, \Theta) = \sum_{Z} q(Z)log \frac{p(X, Z \mid \Theta)}{q(Z)}\]

\[KL(q \mid \mid p) = -\sum_{Z} q(Z)log \frac{p(Z \mid X, \Theta)}{q(Z)}\]

\[\forall q(Z)\]

\(L(q, \Theta)\) - нижняя оценка

Берем \(q(Z) = p(Z \mid X, \Theta)\) - получаем E-шаг

\(L(q, \Theta) = \sum_{Z \sim q(Z)} p(Z) log(...)\) - М-шаг

EM-алгоритм дает гарантии на рост правдоподобия

\section{ЕМ алгоритм 2}

\textbf{Свойства}

\begin{enumerate}
    \item \(logP(X \mid \Theta^{new}) \geq logP(X \mid \Theta^{old})\)
    \item Если $\Theta_{i}$ не является станционарной точкой l, 
    то $\Theta_{i + 1} \neq \Theta_{i}$

    \[\nabla l(\Theta_i) \neq 0\]
    \[logP(X \mid \Theta_i) = L(q \mid \Theta_i) + 
    KL(q(\Theta_i) \mid\mid p)\]
    \[KL = 0 \rightarrow 
    \nabla_{\Theta}KL(q( \mid \mid p) = 0 \rightarrow 
    \nabla L(q \mid \Theta_i) \neq 0 \rightarrow\]
    \[\textrm{На М шаге точно сдвинемся и поменяем }\Theta\]
\end{enumerate}

\textbf{Теорема}

\[Q(\Theta, \Theta^{Old}) = 
E_{z \sim p(Z \mid X, \Theta^{Old})}
logP(Z, X \mid \Theta^{Old})\]

Пусть Q непрерывна по обоим аргументам

Тогда:

\begin{enumerate}
    \item Все предельные точки последовательности $\Theta$
    являются станционарными точками $logP(X \mid \Theta)$
    \item $logP(X \mid \Theta)$ монотонно сходится к 
    $logP(X \mid \Theta^{\star})$ - одной из станционарных точек 
\end{enumerate}

Отвлеченная штука:

X - обучающая выборка

Хотим подогнать под нее распределение $p(x \mid \theta)$

Эмпирическое распределение:

\[\hat{p}(x \mid X) = \frac{1}{l}\sum^{l}_{i = 1}[x = x_i]\]

Минимизировать KL-дивергенцию между эмпирическим и 
параметрическим распределением.

\[KL(\hat{p}(x \mid X) \mid \mid p(x \mid \theta)) 
\rightarrow min_{\theta}\]

\[= \sum_{i = 1}^{l} 
\frac{1}{l}log  \frac{1 / l}{p(x_i \mid \theta)} = 
\sum_{i = 1}^{l} 
\frac{1}{l}log(1 / l) - logp(x_i \mid \theta) \rightarrow \]
\[\sum_{i = 1}{l} logP(x_i \mid \theta) \rightarrow \max_{\theta}\]

\section{Поиск аномалий}

В обучении есть только один класс - неаномальный, 
надо научится отделять от него аномалии

\textbf{Несбалансированная классификация}
\begin{enumerate}
    \item (Under/over)sampling - взвешенный 
    функционал ошибки
    \item Синтетические объекты
    \begin{enumerate}
        \item SMOTE
        \begin{enumerate}
            \item Выбираем объекты $X_1$ из минорного класса,
            выбираем случаный объект из k ближайших соседей
            тоже из минорного класса $X_2$
            \item Новый объект: \(X = \alpha X_1 + (1 - \alpha)X_2, 
            \alpha \sim U(0, 1)\)
            \item Предполагает существование объектов между $X_1, X_2$
        \end{enumerate}
    \item Аугментации
    \end{enumerate}
\end{enumerate}

\textbf{Одноклассовая классификация}

Бенчмарк: Классификация X на нормальные и аномальные, 
стандартные метрики

\begin{enumerate}
    \item Статистический подход - описываем плотностью $p(x)$ для
    новых объектов смотрим на вероятность - $p(x)$ - novelty score.

    Откуда брать p
    \begin{enumerate}
        \item Параметрический подход:
        
        \[\sum_{i = 1}^l logP(x \mid \theta) 
        \rightarrow \max_{\theta}\]
        \item Непараметрический подход:
        \begin{enumerate}
            \item d = 1:
            
            \[p(x) = \lim_{h \to 0} 
            \frac{1}{2h}P(\xi \in [x - h, x + h])\]

            \[\hat{p}(x) = \frac{1}{2h}\frac{1}{l} 
            \sum_{i = 1}^l \mid [x_i - x \mid < h] =\]

            \[= \frac{1}{lh} \sum_{i = 1}^l 
            \frac{1}{2}[\frac{\mid x_i - x \mid}{h} < 1]\]

            Можно заменить на более гладкую плотность:

            \[\frac{1}{lh} \sum_{i = 1}^l 
            \frac{1}{2}K(\frac{x_i - x}{h})]\]

            \begin{enumerate}
                \item $K(z) = K(-z)$
                \item $\int_{\mathcal{R}} K(z)dz = 1$
                \item $K(z) \geq 0$
                \item Не возрастает при $Z > 0$
            \end{enumerate}
            \item d > 1:
            
            \[\hat{p}(x) = \frac{1}{lV(h)} \sum_{i = 1}^l 
            K(\frac{\rho(x_i, x)}{h})\]

            \[V(h) = \int K(\frac{\rho(x_i, x)}{h}) dx\]

            h - гиперпараметр
        \end{enumerate}
    \end{enumerate}

    \item Метрический подход
    
    x - аномалия, если он далеко от других объектов

    Смотреть на количество объектов в $\epsilon$-окрестности?

    Плохой подход:

    Надо смотреть не на единую окрестность, а смотреть на 
    плотность объектов в отдельной точке и на основе нее 
    оценивать окрестность

    \textbf{Определения:}

    \begin{enumerate}
        \item $\rho_k(x)$ - такое минимальное число n, что:
        
        Для $\geq$ k объектов из $X/\{x\}$
        выполнено $\rho(x, z) \leq n$

        Для $\leq$ k-1 объектов выполнено $\rho(x, z) < n$

        По сути: расстоение до k-го ближайшего соседа

        \item К-окрестность: 
        
        \[\mathcal{N}_{k}(x) = \{z \in X/\{x\}\}: 
        \rho(x, z) \leq \rho_k(x)\]

        \item Reachibility Distance:
        
        \[rd_k(x, z) = max(\rho_k(z), \rho(x, z))\]

        Позволяет сгладить расстояние между объектами

        \item Local Reachibility Distance
        
        \[lrd_k(x) = \frac{1}{\frac{1}{\mid\mathcal{N}_{k}(x)\mid}\
        \sum_{z \in \mathcal{N}_{k}(x)}rd_k(x, z)}\]

        Обращенное среднее расстояние от x до ближайших соседей

        \item Local Outlier Factor
        
        \[LOF_k(x) = \frac{\frac{1}{\mid\mathcal{N}_{k}(x)\mid}\
        \sum_{z \in \mathcal{N}_{k}(x)}lrd_k(z)}
        {lrd_k(x)}\]

        Отлавливаем объекты у которых соседи находятся в плотных
        областях, но сами они находятся далеко от соседей
    \end{enumerate}

    \item Model-based AD
    \begin{enumerate}
        \item Есть примеры нормальных объектов
        \item Хотим найти наименьшую область,
        содержащую все объекты

        \[a(x) = sign(<w, x> - \rho)\]

        Идея:

        Отделяем X от начала координат с помощью a()

        \[\begin{cases}
            \frac{1}{2}\mid\mid w \mid\mid^2 + 
            \frac{1}{\nu \ell}\sum \xi_i - \rho
            \rightarrow \min_{w, \xi, \rho} \\
            <w, x_i> \geq \rho - \xi_i, \forall i \\
            \xi_i \geq 0
        \end{cases}\]
        
        $\nu$ - гиперпараметр

        $\sum [a(x) = -1] \leq \nu$

        Требования к решению:

        \begin{enumerate}
            \item Отделить как можно больше объектов от 0.
            За это отвечает $\sum \xi_i$
            \item Максимизировать отступ. 
            За это отвечает $\mid\mid w \mid\mid^{2}$
            \item Гиперплоскость как можно дальше от нуля.
            За это отвечает $\rho$
        \end{enumerate}

        \[a(x) = sign(<w, x> - \rho)\]

        Можно записать двойственную задачу:

        \[
        \begin{cases}
            \frac{1}{2}\sum \lambda_i\lambda_jK(x_i, x_j)
            \rightarrow \min_{\lambda} \\
            0 < \lambda_i \leq \frac{1}{\nu \ell} \\
            \sum \lambda_i = 1
        \end{cases}    
        \]

        С помощью ядра получаем компактную область
    \end{enumerate}

    \item Random projections
    \begin{enumerate}
        \item Isolation Forest
        
        Строим жадное дерево со случайными предикатами по
        случайным признакам

        Если в каком-то листе оказывается 1 объект - прекращаем
        разбиение

        Аномальные объекты рано получают свой лист

        \textit{Обучение:}

        Строим лес из N деревьев, в каждом 
        случайные предикаты. Максимальная глубина
        $D = log_2 \ell$

        \textit{Применение:}

        $h_n(x)$ - оценка аномальности x с точки зрения
        n дерева

        $K_n(x)$ - глубина листа в который попадает x в 
        n дереве

        Нужно сделать поправку на количество объектов в листе

        \[h_n(x) = K_n(x) + C(m_n(x))\]

        \[C(m) = 2H(m - 1) - 2\frac{m - 1}{m}\]

        \[H(i) \approx lni + 0.577\]

        Можно использовать и $log_2(m)$

        \[a(x) = 2^{-\frac{\frac{1}{N}\sum_{n = 1}^{N}h_n(x)}
        {C(l)}}\]

        $C(l)$ - средняя длина пути

        \item Extra Random Trees
        
        Берем индикатор попадания в листья 
        и строим линейную модель
    \end{enumerate}
\end{enumerate}

\textit{Как измерять качество:}

\begin{enumerate}
    \item Anomaly detection
    
    Есть примеры аномалий, но мало данных

    Смотрим какое количество аномалий модель угадывает

    \item Novelty detection
    
    Аномалии не даны, 
    качество модели оценивается глазами
\end{enumerate}

\section{Обучение без учителя}

\begin{enumerate}
    \item Кластеризация: DBScan, 
    Спектральная классификация,
    Affinity Propagation
    \item Внешние метрики качества кластеризации
    \item Тематическое моделирование
\end{enumerate}

\textbf{K-means:}

Основная проблема - ищет сферические кластеры

\subsection{DBScan}

Типы объектов:

\begin{enumerate}
    \item Ядровые:
    
    В $\epsilon$-окрестности находится N объектов

    \item Пограниченые объекты:
    
    Достижимы из ядровых, находится в 
    $\epsilon$-окрестности ядрового

    \item Выбросы:
    
    Все остальные
\end{enumerate}

Псевдокод:

\begin{lstlisting}
    K = 0 #Num clasters
    rho #metric
    epsilon, N #hyperparam
    for i = 1 ... l:
        if label(x[i]) != 0:
            continue
        #point neighborhood
        U = {x in X | rho(x[i], x[j] <= epsilon)}
        if |U| < N:
            label(x[i]) = noise
            continue
        K++ # found new claster if |U| > N
        label(x[i]) = K
        U = U \ {x[i]}
        for x[j] in U:
            if label(x[j]) = noise:
                label(x[j]) = K
            if label(x[j]) != 0:
                continue
            label(x[j]) = K
            #point neighborhood
            R = {xm in X | rho(x[m], x[j]) < epsilon}
            if |R| >= N:
                #new core object neighborhood included in U
                U = U & R
\end{lstlisting}

Преимущества:

\begin{enumerate}
    \item Находит кластеры сложной формы
    \item Находит выбросы
    \item Быстрый
    \item Не надо задавать число кластеров
\end{enumerate}
    
Недостатки:

\begin{enumerate}
    \item Проблемы если кластеры разной плотности
    \item Проблемы с точками на краях
    \item Не работает если кластеры характеризуются неплотность
    \item Не параллелится
\end{enumerate}

\subsection{Иерархическая кластеризация}

Цель: Найти кластерную структуру

Визуализировать разную структуру кластеров при разном 
их количестве

\textit{Агломеративная кластеризация}

Начинаем с того, что каждый объект является кластером

Псевдокод:

\begin{lstlisting}
    C = {{x[1]}, {x[2]}, ..., {x[l]}}
    for n = 2, ..., l:
        G, H = argmin rho(G, H) #find nearest clasters
        C = (C \ {G, H}) U {G U H}    
\end{lstlisting}

Функция расстояния между кластерами $\rho$:

\begin{enumerate}
    \item Single Linkage:
    \[\rho_{sl}(G, H) = 
    \min_{x_i \in G, x_j \in H}\rho(x_i, x_j)\]

    Чувствителен к выбросам

    Главная проблема: Chaining

    Алгоритм подцепляет отдельные объекты, а не кластеры

    Дендрограмма - картинка присоединения объектов

    \item Complete linkage
    
    \[\rho_{cl}(G, H) = 
    \max_{x_i \in G, x_j \in H}\rho(x_i, x_j)\]

    Кластеры не будут компактными

    \item Group Average
    
    \[\rho_{ga}(G, H) = \frac{1}{\mid G \mid \mid H \mid} 
    \sum_{x_i \in G, x_j \in H}\rho(x_i, x_j)\]
\end{enumerate}

\subsection{Графовая кластеризация}

G = (X, E)

E - ребра:

\begin{enumerate}
    \item Полный граф - все вершины связаны
     
    $w_{ij} = exp(-\frac{\norm{x_i - x_j}^2}{2 \sigma^{2}})$
    \item KNN-граф:
    
    $w_ij \neq 0 \Leftrightarrow x_i, x_j$ ближайшие соседи
    
    \item $\epsilon$-граф:
    
    $w_ij \neq 0 \Leftrightarrow \rho(x_i, x_j) \leq \epsilon$ 
\end{enumerate}

\textit{Поиск решение}

\begin{enumerate}
    \item Найти связные компоненты (для 3его варианта)
    
    Тупой метод
    \item Минимальное остовное дерево (Алгоритм Краскала)
    \begin{enumerate}
        \item Начинаем с отдельных вершин
        \item Сливаем два кластера с максимальным ребром между ними
        \item Повторить пока не будет K кластеров
        \item Это агломеративная кластеризация с sl
        \item Решает задачу:
        
        \[\max \min_{x_i \in G, x_j in H} \rho(x_i, x_j)\]
    \end{enumerate}
    \item Спектральная кластеризация
    
    \[A, B \subset X, A \cap B = \emptyset\]
    \[W(A, B) = \sum_{x_i \in A, x_j \in B} w_ij\]
    \[X = A_1 \cup A_2 \cup \ldots A_k\]

    Ошибка кластеризации:

    \[\textrm{Ratio Cut}(A_1, \ldots, A_k) = 
    \frac{1}{2}\sum_{i = 1}^K \frac{w(A_i, \bar{A}_i)}
    {\mid A_i \mid} \rightarrow \min_{A_i, \ldots, A_k} (\star)\]
    \[\bar{A}_i = X \setminus A_i\]

    Хотим, чтобы ребра между кластерами были как можно менее
    значимы $\rightarrow$ Каждый кластер должен быть изолированным

    $K = 2 \rightarrow$ Задача поиска максимального потока

    $K > 2 \rightarrow$ NP-полная задача

    \rule{\linewidth}{0.5pt}

    $G = (X, E)$
    
    $d_i = \sum_{j = 1}{d} w_ij$ - сумма ребер, 
    которые с ней связаны

    $D = diag(d_1, \ldots, d_l)$

    $L = D - W$, W - матрица смежности, L - \textbf{лаплассиан}

    \textit{Свойства лаплассиана}

    \begin{enumerate}
        \item \[f \in R^{d}\]
        \[f^TLf = \frac{1}{2}\sum_{i, j = 1}^l 
        w_{ij}(f_i - f_j)^2\]
        \item L - симметричная
        \item L - неотрицательно определенная
    \end{enumerate}

    \textbf{Теорема}

    \begin{enumerate}
        \item Кратность $\lambda = 0$ у L равна
        числу компонент связности графа
        
        Кратность:
        
        \begin{enumerate}
            \item Собственные значения: $Ax = \lambda x$
            \item $det(A - \lambda I) = 0 \rightarrow \lambda_i$
            \item $\lambda_i$ - решениe
            \item ${\lambda_i}, \forall i$ - спектр графа
            \item Характеристическое уравнение выражаем в виде
            характеристического многочлена и раскладываем его
            в виде решений

            \[P_A(\lambda) = 
            (-1)^{n}\prod_{i = 1}^{n}(\lambda - \lambda_i) = 
            (-1)^n(\lambda - \lambda_1)^{k_1}\ldots\]

            $k_1$ называется кратностью для $\lambda_1$
        \end{enumerate}
        \item $A_1, \ldots, A_k$
        
        Вектор индикатор: $f_i = ([x_j \in A_i])_{j = 1}^{\ell}$

        $f_1, \ldots, f_k$ - собственные векторы для $\lambda = 0$
    \end{enumerate}

    \underline{Доказательство:}

    K = 1:

    \begin{enumerate}
        \item Является ли $\lambda = 0$ собственным значением
        \[f = (1, \ldots, 1)\]
        \[Lf = Df - Wf =
        \begin{pmatrix}
            d_1 \\
            \vdots \\
            d_{\ell}
        \end{pmatrix} - 
        \begin{pmatrix}
            \sum w_{1, j} \\
            \vdots \\
            \sum w_{\ell, j}
        \end{pmatrix} = 0\]
        \item Кратность $\lambda = 0$ = 1 $\rightarrow$
        нет других собственных векторов
        
        Допустим:
        \[\exists f^{\prime} \in R^{\ell}: 
        \exists p \neq q,
        f^{\prime}p \neq f^{\prime}q, 
        Lf^{\prime} = 0\]

        \[(f')^TLf' = \frac{1}{2}\sum_{i, j = 1}^{\ell}
        w_{ij}(f_i' - f_j')^2 = 0\]
        \[\forall i,j: 
        \bcases{w_{ij} = 0 \textrm{ Нет ребра} \cr 
        f_i' = f_j'}\]

        Граф G связный $\rightarrow$ 
        Существует путь из p в q $\rightarrow$

        Путь: $w \rightarrow i_1 \rightarrow 
        \ldots \rightarrow p$

        $w_{pi_1} \neq 0 \rightarrow f'p = f'_{i_1}
        \rightarrow w_{i_1\ldots} \neq 0 \rightarrow \ldots
        \rightarrow f'_p = f'_{i_1} = \ldots = f'_q$

        \[\bot\]
    \end{enumerate}

    K > 1:

    Можно упорядочить объекты так, чтобы L был
    блочно-диагональным
    
    \[L = 
    \begin{pmatrix}
        L_1 & 0 & 0 \\
        0 & L_2 & 0 \\
        & \ddots & \\
        0 & 0 & L_K
    \end{pmatrix}\]

    Спектр блочно-диагональной матрицы = 
    объединение спектров отдельных блоков

    $L_i \rightarrow f_i = ([x_j \in A_i])_{j = 1}^{\ell}$

    Кратность $\lambda = 0$ равна K

    \rule{\linewidth}{0.5pt}

    Гипотеза: $x_i, x_j$ - похожие объекты $\Rightarrow$
    у собственного вектора $f_i$ для $\lambda_i \approx 0$,
    будет $f_ij \approx f_ik$

    Для связанных графов не берем $\lambda = 0$, 
    т.к. тогда будет одна компонетна

    \textbf{Алгоритм:}
    \begin{enumerate}
        \item Строим лаплассиан
        \item Находим m (гиперпараметр) нормированных 
        собственных векторов $u_1, \ldots, u_m$,
        соотв. наименьшим собственным значениям. 
        
        Сложность: $O(l^3)$
        \item $U = (u_1 \mid \ldots \mid u_m) \in R^{\ell \times m}$
        \item Новые признаки близки для объектов 
        в одной плотной области
        \item K-means
    \end{enumerate}

    \textbf{Как это связано с задачей $(\star)$:}

    Если эту задачу релаксировать и 
    искать не жесткое приписывание к классам, а распределение,
    то ее решение U.
\end{enumerate}
\pagebreak

\part{Семинары}
\section{Семинар: Задачи условной оптимизации}

\textit{Учебник: Boyd, Convex Optimization}

\[\begin{cases}
    f_0(x) \rightarrow min_{x \in R^d} \\
    f_i(x) \leq 0, i = 1, \ldots, m \\
    h_i(x) = 0, i = 1, \ldots, p
\end{cases}\]

\[G(x) = f_0(x) + \sum_{i = 1}^m I_{-}(f_i(x)) + \sum_{i = 1}^p I_0(h_i(x)) \rightarrow min\]

Штрафы за нарушение ограничений:

\[I_{-}(z) = \begin{cases}
    0, z \leq 0 \\
    + \infty, z > 0 
\end{cases}\]

\[I_{0} = \begin{cases}
    0, z = 0 \\
    + \infty, z \neq 0 
\end{cases}\]

\(G(x) \rightarrow \infty\) в точках где не выполняется условие

Проблема: Недифференцируема

Заменяем функции на их аппроксимации ($\hat{I}_{-} = ax$)

Лагранжиан:

\[L(x, \lambda, \nu) = f_0(x) + \sum_{i = 1}^m \lambda_i f_i(x) + \sum_{i = 1}^p \nu_i h_i(x)\]
\[\lambda_i \geq 0\]

x - прямые (primal) переменные

$\lambda, \nu$ - двойственные переменные

\textbf{Двойственная функция}

\[g(\lambda, \nu) = \inf_{x} L(x, \lambda, \nu)\]

\begin{itemize}
    \item Двойственная функция всегда вогнутая
    \item Дает нижнюю оценку на минимум функции в прямой задаче
    
    \(x^{\prime}\) - допустимая точка (все условия выполнены)

    \[L(x^{\prime}, \lambda, \nu) = f_0(x^{\prime}) + 
    \sum_{i = 1}^m \lambda_i f_i(x^{\prime}) + \sum_{i = 1}^p \nu_i h_i(x^{\prime})\]
    \[f_i(x) \leq 0, h_i(x) = 0 \rightarrow 
    L(x^{\prime}, \lambda, \nu) \leq f_0(x^{\prime})\]
    \[\inf_x L(x, \lambda, \nu) \leq \inf_{x^{\prime}} 
    L(x', \lambda, \nu) \leq \inf_{x^{\prime}} f_0(x')\]

    $\uparrow$ - это и есть решение исходной задачи

    \[g(\lambda, \nu) \leq f_0(x_{\star})\]

    \[g(\lambda, \nu) \rightarrow \max_{\lambda, \nu}, \lambda_i \geq 0\]

    \(\lambda^{\star}, \nu^{\star}\) - решение двойственной задачи

    \(g(\lambda^{\star}, \nu^{\star}) \leq f_0(x_{*})\) - слабая двойственность

    \(g(\lambda^{\star}, \nu^{\star}) = f_0(x_{*})\) - сильная двойственность

    \underline{Достаточное условие сильной двойственности (Условие Слейтера)}

    \begin{itemize}
        \item Задача выпуклая:
        
        \(f_0, f_1, \ldots, f_m\) - выпуклые

        \(h_1, \ldots, h_p\) - линейные
        \item \(\exists x^{\prime}\), что все ограничения выполнены строго
    \end{itemize}
\end{itemize}

Пусть имеет место сильная двойственность:

\[g(\lambda^{\star}, \nu^{\star}) = f_0(x_{*})\]

\[g(\lambda^{\star}, \nu^{\star}) = \inf_x(f_0(x) + 
\sum \lambda^{\star} f_i(x) + \sum \nu^{\star} h_i(x))
\leq f_0(x_{\star}) + 
\sum \lambda^{\star} f_i(x_{\star}) + \sum \nu^{\star} h_i(x_{\star}) 
\leq f_0(x_{\star})\]

Все неравенства являются равенствами:

\begin{itemize}
    \item Если решить безусловную задачу при подставлении $\lambda^{\star}, \nu^{\star}$,
    то получим решение прямой задачи
    \item \(\lambda_i^{\star}f_{i}(x^{\star}) = 0\) - условие дополняющей нежесткости
\end{itemize}

\textbf{Теорема Куна-Такера}

Необходимые условия для 

\[
\begin{cases}
    \nabla_x L(x_{\star}, \lambda^{\star}, \nu^{\star}) = 0 \\
    f_i(x) \leq 0 \\
    h_i(x) = 0 \\
    \lambda_i \geq 0 \\
    \lambda_i f_i(x_{\star}) = 0 \\
    \textrm{Сильная двойственность}
\end{cases}
\leftrightarrow x_{\star}, \lambda^{\star}, \nu^{\star} \textrm{решения}
\]

\section{Семинар 3: EM алгоритм}

На М-шаге:

\[\Theta = \argmax_{\Theta} E_{q} log p(X, Z \mid \Theta)\]

\[log p(X \mid \Theta_{i + 1}) > log p(X \mid \Theta_{i})\]


Задача: \textbf{Шумная разметка изображений 100 экспертами}

i - изображение, j - эксперт: $l_{ij} \in \{0, 1\}$

Истинный класс для картинки $Z_i \in \{0, 1\}$

Дополнительные параметры:

\[\beta_i \in (0, +\infty), \alpha_j \in \mathcal{R}\]

$\beta$ - сложность изображения, $\alpha$ - уровень эксперта

\[p(l_{ij} = Z_i \mid Z_i, \alpha_j, \beta_i) = \sigma(\alpha_j \beta_i) 
= \frac{1}{1 + e^{-\alpha_j \beta_i}}\]

\[p(Z_i, l_i \mid \alpha, \beta) = 
p(Z_i) \prod_{j} p(l_ij \mid Z_i, \alpha_j, \beta_i)\]

\(p(Z_i)\)?

\begin{enumerate}
    \item Задать как 1/2, т.к. имеем два класса
    \item Задать как баланс классов
    \item Найти как параметр \(p(1) = \pi\)
\end{enumerate}

\[p(Z, l \mid \alpha, \beta) = 
\prod_i Z_i \prod_{j} p(l_ij \mid Z_i, \alpha_j, \beta_i)\]

Необходимо свести вероятность $l_{ij} = Z_i$ к вероятности $l_{ij}$

\[p(l_{ij} = Z_i \mid \ldots) = \sigma(\alpha\beta)\]

\[p(l_{ij} \neq Z_i \mid \ldots) = 1 - \sigma(\alpha\beta)\]

Бернулли:

\[p(l \mid \ldots) = p(l = Z \mid \ldots)^{[l = Z]} 
\times p(l \neq Z \mid \ldots)^{[l \neq Z]} =
\sigma(\alpha\beta)^{[l = Z]}\sigma(-\alpha\beta)^{[l \neq Z]}\]

\[p(Z_i, l_{ij} \mid \ldots) = 
p(Z_i)\prod_{j}\sigma(\alpha\beta)^{[l = Z]}
\sigma(-\alpha\beta)^{[l \neq Z]}\]

\textbf{Е-шаг:}

\[q^{\star}(Z_i) = p(Z_i \mid l_{ij}, \alpha_j, \beta_i) 
\xrightarrow{\textrm{Теорема Байеса}} \frac{p(Z_i, l_{ij}\mid\alpha,\beta)}
{p(l_{ij}\mid\alpha,\beta)} = \frac{p(Z_i, l_{ij}\mid\alpha,\beta)}
{\sum_{t} p(t, l_{ij}\mid\alpha,\beta)}\]

\[q^{\star}(Z) = \frac{p(Z_i)
\prod_{j}\sigma(\alpha\beta)^{[l = Z]}
\sigma(-\alpha\beta)^{[l \neq Z]}}
{\sum_{t \in \{0, 1\}} p(t_i)
\prod_{j}\sigma(\alpha\beta)^{[l = t]}
\sigma(-\alpha\beta)^{[l \neq t]}} 
= \frac{\gamma^{Z_i}_i}{\gamma^{0}_i + \gamma^{1}_i} = 
\frac{e^{log\gamma^{Z_i}_i}}
{e^{log\gamma^{0}_i + \gamma^{1}_i}}\]

\textbf{M-шаг:}

\[E_{q^{\star}} log p(Z, l \mid \alpha, \beta) 
\rightarrow \max_{\alpha, \beta}\]

\[E_{q^{\star}} log \prod_i p(Z, l \mid \alpha, \beta) = 
\sum_i E_{q^{\star}_i} logp(Z, l \mid \alpha, \beta) = \]
\[=\sum_i E_{q^{\star}_i}[logp(Z_i) + 
\sum_j [l_{ij} = Z_i] log \sigma(\alpha\beta) + 
[l_{ij} \neq Z_i] log \sigma(-\alpha\beta)] 
\rightarrow \max_{\alpha, \beta}\]

\[\sum_i \sum_j \sum_{t \in \{0, 1\}} 
q_i^{\star}(t)[[l_{ij} = t] log \sigma(\alpha\beta) + 
[l_{ij} \neq t] log \sigma(-\alpha\beta)]\]

Оптимизируем:

\[\frac{\partial}{\partial x} log\sigma(x) = \sigma(-x)\]

\[\frac{\partial}{\partial \alpha}
log\sigma(\alpha\beta)= 
\beta\sigma(-\alpha\beta)\]

\[\frac{\partial}{\partial \alpha} \log\sigma(-\alpha\beta) = 
-\beta\sigma(\alpha\beta)\]

\[\frac{\partial}{\partial \alpha} E_{q^{\star}} 
log p(Z, l \mid \alpha, \beta) = 
\sum_i \sum_t q_i^{\star} \beta ([l = t]\sigma(-\alpha\beta)) - 
[l \neq t]\sigma(\alpha\beta))\]

По $\beta$ аналогично

\section{Семинар 4: Основы байсовских методов}

Существует распределение $p(x, y)$

Интересует распределение: $p(y \mid x)$

\textit{Формула Байеса}

\[p(y \mid x) = \frac{p(x \mid y)p(y)}{p(x)}\]

$p(x \mid y)$ - правдоподобие X, распределение объектов
для некоторого класса

$p(y)$ - априорное распределение, 
доли классов в обучающей выборке

$p(x)$ - нормировочная константа

\textit{Функционал среднего риска}

\[R(a) = \int_Y \int_X L(y, a(x))p(x, y)dxdy\]

\[E_{y, x}L(y, a(x))\]

\textit{Как использовать оптимальное распределение, 
когда оно найдено?}

\[L(y, a) = [y \neq a]\]

Функционал среднего риска:

\[R(a) = \int_Y \int_X [y \neq a(x)]p(x, y)dxdy 
= \sum_{y = 1}^{K}\int_X[y \neq a(x)]p(x, y)dxdy =\]
\[= \int_X \sum_{y \neq a(x)}p(x, y)dxdy 
= \int_X (1 - \sum_{y = a(x)}p(x, y))dxdy =\]
\[1 - \int_X p(x, a(x))dxdy \rightarrow \min \Rightarrow
a_{\star}(x) = \argmax_{y \in Y} P(y \mid x)\]

Для регрессии:

\[L(y, a) = (y - a)^2\]
\[a_{\star}(x) = E(y \mid x)\]

\textit{Как найти $p(y \mid x)$}

В классификации: 

\[a_{\star}(x) = 
\argmax_{y \in Y} p(y \mid x) = 
\argmax_{y \in Y} \frac{p(x \mid y)p(y)}{p(x)} = \]
\[= \argmax_{y \in Y}p(x \mid y)p(y)\]

$p(y)$ задается исходя из распределения $y$

$p(x \mid y, \theta)$ находим $\theta$ ММП
\newline

\textit{Пример:}

\rule{\linewidth}{0.5pt}
\[p(y \mid x, w) = \mathcal{N}(<w, x>, \sigma^{2})\]

Правдоподобие:

\[\prod_{i = 1}^{\ell}\frac{1}{\sqrt{2 \pi \sigma^{2}}}
exp(-\frac{(y_i - <w, x_i>)^{2}}{2\sigma^{2}}) 
\rightarrow \max_{w}\]

\[logL = -\ell log\sqrt{2 \pi \sigma^{2}} - 
\frac{1}{2 \sigma^{2}}\sum_{i = 1}^{l}(y_i - <w, x_i>)^2
\rightarrow \max_{w} \Rightarrow\]
\[\Rightarrow \sum_{i = 1}^{l}(y_i - <w, x_i>)^2 
\rightarrow \min_{w}\]
\rule{\linewidth}{0.5pt}
\newline

Классификация:

Нужно найти $p(x \mid y, \theta)$ для всех классов

\[p(x \mid y, \theta) = \mathcal{N}(\mu_{y}, \Sigma_{y})\]

Можем найти $\mu_y, \Sigma_y$ по ММП

Если параметры распределены нормально - 
Нормальный дискриминантный анализ

Если $\Sigma_y = \Sigma$, метод называется
линейный дискриминант Фишера

Разделяющая поверхность:

\[p(y = +1 \mid x, \theta) = p(y = -1 \mid x, \theta)\]

$\Sigma_{-1} \neq \Sigma_{+1} \Rightarrow$ квадратичная поверхность

$\Sigma_{-1} = \Sigma_{+1} \Rightarrow$ Линейная поверхность

\rule{\linewidth}{0.5pt}

\textbf{Больше распределений:}

\[p(w \mid y, x) = \frac{p(y \mid x, w)p(w)}{p(x, y)}\]

$p(w) \sim \mathcal{N}(0, \sigma^{2}I)$ - запрещаем модели большие веса

\[logP(w \mid y, x) = 
-\frac{1}{2\sigma_{2}}\sum_{i = 1}^{\ell}(y_i - <w, x_i>)^2 -
\frac{\ell}{2\alpha^{2}}\sum_{j = 1}^{\alpha}w_{j}^{2} 
\rightarrow \max_{w}\]

Фактически: MSЕ с регуляризацией $L^2$

\[\lambda = \frac{\ell \sigma^{2}}{\alpha^{2}}\]

Что если $w_j \sim \mathcal{0, \alpha_j^2}$?

Отдельный коэффициент регуляризации для каждого
параметра - такое не особо выводится в классическом 
машинном обучении $\rightarrow$ RVM

\rule{\linewidth}{0.5pt}

\textbf{Наивный Байесовский алгоритм}

Исходя из предположения о независимости признаков:

\[p(x \mid y) = \prod_{j = 1}^{d}p(x_j \mid y)\]
\[a(x) = \argmax_{y \in Y}p(y \mid x) = 
\argmax_{y}(lnP(y) + \sum_{j = 1}^d lnP(x_j \mid y))\]
\end{document}
